\documentclass[a4paper]{article}
 
\usepackage{fullpage}
\usepackage{titlesec}
\usepackage[utf8]{inputenc}
\usepackage[T1]{fontenc}
\usepackage[french]{babel}
\usepackage[autolanguage]{numprint}
\usepackage{hyperref}
\usepackage[a4paper]{geometry}

 
\title{Architecture des ordinateurs\\\large Projet : Jeu du Snake}
\author{BEAUREPERE Mathias, GALL Léon}
\date{}
 
\hypersetup{
  pdftitle={Snake},
  pdfauthor={BEAUREPERE Mathias,GALL Léon},
}
\geometry{
 a4paper,
 left=15mm,
 top=10mm,
 right=15mm,
 bottom=20mm,
 }

\makeatletter

% % % % % % % % % % % % % % % % % % % % % % % % % % % % % % % % % % % % % % 
%                                                                         %
%                                 DOCUMENT                                %
%                                                                         %
% % % % % % % % % % % % % % % % % % % % % % % % % % % % % % % % % % % % % % 
\begin{document}

% % % % % % % % % % % % % % % % % % % % % % % % % % % % % % % % % % % % % % 
%                                 HEADER                                  %
% % % % % % % % % % % % % % % % % % % % % % % % % % % % % % % % % % % % % % 
\hspace{-0.5cm}\begin{minipage}{0.5\textwidth}
Université de Strasbourg\\
UFR de mathématique et d'informatique
\end{minipage}
\hspace*{\fill}\begin{minipage}{0.5\textwidth}
\hspace*{\fill}L2, Semestre 3, Automne 2021\\
\hspace*{\fill}\@author
\end{minipage}
\\
\begin{center}
  \huge \@title
\end{center}
\vspace*{1cm}

% % % % % % % % % % % % % % % % % % % % % % % % % % % % % % % % % % % % % % 
%                                 RAPPORT                                 %
% % % % % % % % % % % % % % % % % % % % % % % % % % % % % % % % % % % % % % 

Le projet d'architecture des ordinateurs était l'implémentation du célèbre jeu snake en assembleur MIPS.
Pendant la conception, nous avons dû faire des choix.

% % % % % % % % % % % % % % % % % % % % % % % % % % % % % % 
%                  CHOIX D'IMPLEMENTATION                 %
% % % % % % % % % % % % % % % % % % % % % % % % % % % % % %
\section{Choix d'implémentations}

% % % % % % % % % % % % % % % % % % % % % 
%                  STYLE                %
% % % % % % % % % % % % % % % % % % % % %
\subsection{Style}
Nous avons choisi  de reprendre le style de programmation de la partie du projet fournie en amont par les professeurs.
C'est pourquoi nous avons mis des indentations uniquement pour les boucles, et avons utilisé des virgules
pour séparer les registres dans les instructions.

Nous nous sommes conformés aux normes MIPS, en utilisant les registres sauvegardés
dans nos fonctions appelant d'autres fonctions en interne (en les sauvegardant en amont dans le prologue puis en les restaurant dans l'épilogue).
Dans les fonctions n'appelant pas de fonctions ou pour des informations non réutilisées après l'appel d'une fonction,
nous avons utilisés les registres temporaires.

% % % % % % % % % % % % % % % % % % % % % 
%         STRUCTURES DE DONNEES         %
% % % % % % % % % % % % % % % % % % % % %
\subsection{Structures de données}

% % % % % % % % % % % % % % % % % % % % % 
%               FONCTIONS               %
% % % % % % % % % % % % % % % % % % % % %
\subsection{Fonctions}

% % % % % % % % % % % % % % % % % % % % % 
%       FONCTIONNALITES AJOUTEES        %
% % % % % % % % % % % % % % % % % % % % %
\subsection{Fontionnalités ajoutées}

% % % % % % % % % % % % % % % % % % % % % % % % % % % % % % 
%    REPARTITION DU TRAVAIL ET DIFFICULTES RENCONTREES    %
% % % % % % % % % % % % % % % % % % % % % % % % % % % % % %
\section{Répartition du travail et difficultés rencontrées}

% % % % % % % % % % % % % % % % % % % % % 
%        REPARTITION DU TRAVAIL         %
% % % % % % % % % % % % % % % % % % % % %
\subsection{Répartition du travail}
Nous avons choisi de nous répartir les fonctions. 

% % % % % % % % % % % % % % % % % % % % % 
%        DIFFICULTES RENCONTREES        %
% % % % % % % % % % % % % % % % % % % % %
\subsection{Difficultés rencontrées}

% % % % % % % % % % % % % % % % % % % % % % % % % % % % % % 
%                       OBSERVATIONS                      %
% % % % % % % % % % % % % % % % % % % % % % % % % % % % % %
\section{Observations}
%  ...

\end{document}